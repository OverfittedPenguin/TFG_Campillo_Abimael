\chapter{Conclusions and Future Work}

\section{Main Contributions}
In this project, an algorithm has been developed for generating optimal trajectories in firefighting manoeuvres, fulfilling the primary objective of this thesis. The methodology successfully synthesised Optimal Control Theory (\gls{oct}) and Non-linear Programming (\gls{nlp}) techniques, employing a Direct Collocation Method (\gls{dcm}) and a trapezoidal integration scheme to model generic firefighting operations for unmanned aerial systems. This theoretical framework was implemented using the CasADi optimisation suite to develop an IPOPT-based solver, which consistently yielded feasible and optimal solutions tailored to the operational parameters of the Flyox \gls{uas}.

Firstly, a reduced version of a cruise flight has been modelled and implemented, enabling the validation of the acquired knowledge about \gls{oct} and \gls{nlp}. Subsequently, the requirements for the extinguishing manoeuvre for a \gls{uav} were identified and mathematically modelled. After mathematically modelling the extinguishing manoeuvre in three different phases -descent, cruise with water discharge, and ascent- various simulations were carried out to observe the criticality of the integration as well as the operability of the Flyox \gls{uas} under optimal control. 

Essentially, it has been possible to show that the integrated system has a high degree of robustness against possible instantaneous disturbances. This verification was carried out by measuring the opening time of the gates through which water is discharged onto the flames. When the system was disturbed by a sudden decrease in mass, the algorithm was able to find an optimal and feasible solution to compensate for this effect minimising the variation in control and states. However, under nominal operating conditions -a longer discharge regime-the control inputs are smoother and result in flatter state trajectories. Therefore, it is confirmed that the algorithm or solver maintains stability under certain stress, although further testing should be carried out.

Subsequently, a pivot analysis was performed on the effect of the end-time within the terminal cost. While a heavily penalised free end-time reduces the total duration of the manoeuvre -critical in forest firefighting manoeuvres- it leads to more spiky control trajectories and, therefore, less smooth state profiles. However, despite this aggressiveness, including the end-time penalty helps to facilitate transitions between stages, suggesting that for firefighting operations, this penalty should be balanced. The aim is to prioritise speed while maintaining safety and efficiency in all stages.

Then, by easing constraint margins and reducing the penalisation in state deviations, a significant improvement in control profile continuity has been observed. This phenomenon is mathematically grounded in the expansion of the \gls{kkt} optimality margins, particularly regarding dual feasibility. Specifically, the solver finds solutions which can move away from the wall of constraints while generating greater optimal control. Nevertheless, it must be said that this has been possible because, at the same time as the control parameters' effect was reduced in the cost functional, constraint margins were larger; if the margins had not been extended, the optimality margin would probably have decreased.  This highlights that in high-dimensional \gls{nlp} rigid constraint boundaries can often hinder the search for truly optimal control, whereas broader margins allow the \gls{ipopt} solver to find solutions with better state profiles.

In fourth place, the comparative study between different objective schemes underscores the necessity of a well-defined cost functional. Simulations relying solely on terminal costs revealed a tendency toward bang-bang control behaviour, which triggers undesirable state oscillations as the system nears its constraint limits. Conversely, a constraint-only approach proved insufficient for practical application; while it yields feasible solutions, the absence of an objective goal results in unnecessarily long and aggressive manoeuvres. These findings confirm that for a platform like the Flyox \gls{uas}, the objective function must be a multi-variable compromise that motivates the solver not just to find a solution, but to find the smoothest possible path.

In conclusion, this project establishes a robust mathematical and computational baseline for the autonomous operation of the Flyox \gls{uas} in firefighting missions. While the current 2D point-mass model and environmental simplifications represent a necessary starting point for stability analysis, the successful implementation of the IPOPT-based solver proves that non-linear programming is a viable path for real-time trajectory generation. In addition, this work has provided a framework that enhances both the efficiency and safety of aerial firefighting, laying the groundwork for future integration into high-fidelity environments.

\section{Limitations and Future Research}

The developed algorithm is far from being the perfect solution to the problem, as it has some limitations in terms of integration and development, as listed below.

\begin{itemize}
    \item \textbf{Bi-dimensional Approach.} The current algorithm has been planned to solve bi-dimensional (2D) vertical plane restricted trajectories; nevertheless, real-world scenario includes lateral stability effects or dynamics, which is a limitation in firefighting manoeuvres.
    \item \textbf{Environmental Simplification.} All atmospheric conditions, terrain elevation or obstacles have been neglected or simplified to offer a quick solution to the problem but real scenarios does not have those simplifications. They are critical elements that have to be considered in low-altitude flight manoeuvres.
    \item \textbf{Point-Mass Dynamics.} The mathematical formulation and implementation behind the algorithm has reduced a real aircraft to a point-mass model, simplifying all the dynamics and kinematics.
    \item \textbf{Dynamic Stability and Control Feed-Back.} No dynamic stability has been considered during the algorithm's development, which means that all transitory have been negelected because there is no control feed-back.
    \item \textbf{Real-Time Computation.} The algorithm has been developed in Python, oferring a sufficient solution to the needed base. Nonetheless, a real-world mission would need a faster and embedded algorithm, which implies migration to other codes that have CasADi optimisation libraries or similars.
\end{itemize}

Eventually, the following aspects have to be treated in the near future.

\begin{itemize}
    \item \textbf{Integration of Data Elevation Models (\gls{dem}), No-Fly Zones (\gls{nfz}) and Real Atmospheric Conditions}. By using pre-loaded or on-board data, the obstacles and the terrain shall be converted into updated constraints of the optimal control algorithm. In addition, the atmosphere effects shall be included in real trajectory computation.
    \item \textbf{Stochastic Dynamic Modelisation.} A real aircraft dynamic stability should be modelled including stationary and transitory stability remarks and effects, including a stochastic optimal control approach that deals with non-ideal disturbances atmospheric conditions.
    \item \textbf{On-Board Real-Time Computation.} An embedded solution is the best approach to the firefighting scenario, since the manoeuvres are performed in a very dynamic environment. Within this context, a communication protocol between the output of the algorithm and the \gls{uav} autopilot shall be included by waypoint generation or other specifical needs.
\end{itemize}