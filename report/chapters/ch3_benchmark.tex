\chapter{Benchmark Problem: A Level-Flight Cruise Trajectory}
\textcolor{red}{This chapter introduces the reader to flight dynamics using a level-cruise flight trajectory. Also, that trajectory is formulated as an \gls{ocp} and solved via an implementation using CasADi. Therefore, a mathematical formulation on \gls{ocp} notation is presented right after the equations of motion and, right before presenting the results. Eventually, the free-end condition is applied to the problem, acting as a preamble of the following chapter.}

\section{UAV Kinematics and Equations of Motion}
\label{sec:EOM}
Firstly, the flight dynamics for a general \gls{uav} flight trajectory had to be introduced and described. Since the aim of the project is \gls{oct} development for firefighting manoeuvres, some hypothesis have to be applied. The hypotheses considered are the follwoing ones.

\begin{itemize}
    \item A bidimensional (2D) trajectory without tridimensional (3D) effects will be considered. Thus, all lateral stability and movements will be neglected and only longitudinal static stability will be considered -i.e $\mathcal{L} = 0$, $\mathcal{N}=0$, $\Phi = 0$, $\dot{\Phi} = 0$, $\Psi = 0$, $\dot{\Psi} = 0$, $y = const.$ and $\dot{y} = 0$-.
    \item All dynamic modes are neglected and only static stability phenomena will be included.
    \item The atmospheric condistions shall be modelled using International Standard Atmoshpere (ISA) model. Only static bidimensional (2D) wind field could be sonsidered, which means that free-stream velocity has to be constant in all domain.
    \item Bidimensional (2D) level-flight cruise trajectory implies a nul flight path angle and nul variation of the same -i.e $\gamma = 0$ and $\dot{\gamma} = 0$-.
\end{itemize}

Once the hypostheses have been written, its time to introduce the flight dynamics that are involved in bidimensional (2D) flight trajectories. The image below shows the free body diagram  and the relation between plane reference systems -i.e including body axes ($x_b$,$z_b$), wind axes ($x_w$,$z_w$) and local-horizon axes ($x_h$,$z_h$)-.

Applying Newton's second law, the equilibrium of forces and moments can be obtained. To see the different elements involved in each equation, the reader can refer to nomenclature's section. 

\begin{equation}
    \sum F_{x_b} = m\frac{du}{dt} = m\dot{u} = T + L\sin(\alpha) - D\cos(\alpha) - W\sin(\theta) - mqw 
    \label{eq:Forces_Xb}
\end{equation}

\begin{equation}
    \sum F_{z_b} = m\frac{dw}{dt} =m\dot{w} = W\cos(\theta) - L\cos(\alpha) - D\sin(\alpha) + mqu
    \label{eq:Forces_Zb}
\end{equation}

\begin{equation}
    \sum M_{y_b} = I_{y}\frac{dq}{dt} =I_{y}\dot{q} = C_m\bar{q}S\bar{c} - T\left[\Delta X_{CT}\sin(\varepsilon) + \Delta Z
    _{CT}   \cos(\varepsilon)\right]
    \label{eq:Moment_Yb}
\end{equation}

Along to the dynamic equations, the \gls{uav} kinematics can be described using the rotation matrix between body and horizontal axes as follows.

\begin{equation}
    \begin{bmatrix}
        \dot{x} \\
        \dot{z}
    \end{bmatrix}
    \coloneqq R_{hb}
    \begin{bmatrix}
        u\\
        w
    \end{bmatrix}
    =
    \begin{bmatrix}
        \cos \theta & \sin \theta\\
        -\sin \theta & \cos \theta
    \end{bmatrix}
    \begin{bmatrix}
        u \\
        w
    \end{bmatrix}
    = 
    \begin{bmatrix}
       u \cos \theta + w \sin \theta\\
        -u \sin \theta + w \cos \theta
    \end{bmatrix}
    \label{eq:Inertial_Velocities}
\end{equation}

Eventually, pitch and mass variations have to be considered to complete the full dynamic and kinematics set of equations. It has to be pinpointed that for this benchmark problem, only specific fuel consuption is considered on mass variation equation.

\begin{equation}
    \frac{d\theta}{dt} \coloneqq \dot{\theta} = q
    \label{eq:Theta_dot}
\end{equation}

\begin{equation}
    \frac{dm}{dt} \coloneqq \dot{m} = -SFC
    \label{eq:Mass_dot}
\end{equation}

As a clarification on the equation terms, below are the complete formulation of aerodynamic forces and moments, with all proper aerodynamic coefficients considered. 

\begin{equation}
    L \coloneqq \frac{1}{2}\bar{q}SC_L = \frac{1}{2}\bar{q}S\left(C_{L,0} + C_{L,\alpha}\alpha + C_{L,\delta_e}\delta_e\right)
    \label{eq:Lift}
\end{equation}

\begin{equation}
    D \coloneqq \frac{1}{2}\bar{q}SC_D = \frac{1}{2}\bar{q}S\left[C_{D,0} + k\left(C_{L,0} + C_{L,\alpha}\alpha + C_{L,\delta_e}\delta_e\right)^2\right]
    \label{eq:Drag}
\end{equation}

\begin{equation}
    C_m \coloneqq C_{m,0} + C_{m,\alpha}\alpha + C_{m,\delta_e}\delta_e
    \label{eq:Moment}
\end{equation}

\section{Mathematical Formulation of the OCP}
Next step is to transform the problem above into a set of equations using the \gls{oc} notation. To do so, the three phases of optimisation problems descriptions is used, which means that the problem has to be modelled, constrained and then optimised.

\subsection{States and controls}
First step is modelisation, which means that state and control variables have to be defined for the problem. Since the benchmark problem is a level-flight cruise trajectory, the states involved in the previous equations of motion are: local-horizon position ($x$,$z$); body velocities ($u$,$w$); pitch ($\theta$); pitch rate ($q$); and mass ($m$). Those seven states are the ones that allow describing the trajectory and \gls{uav} condition through time, since they indicate position, attitude and mass.

The previous states are defined for a bidimensional (2D) frame using only longitudinal stability equations, as has been indicated before. Therefore, the controls that shall be used are no other ones but \gls{tps}, denoted as $\delta_{TPS}$, and elevator deflection, denoted as $\delta_e$. That pair of control variables are the minimum controllable conditions for the longitudinal flight given the conditions of the problem and they are responsible of controlling speed and altitude, respectively -i.e \gls{uav} speed is controlled by variations in \gls{tps} while \gls{uav} altitude is controlled by $\delta_e$ variations-.

Eventually, the state and control vector can be constructed, followed by its form using state space notation -where $x_i$ are states and $u_i$ are controls-. Notice that the order followed to construct the vector has to be states variables followed by control variables; nevertheless, is indistinct which state or control goes first in each group. The order has been chosen by convenience for later \gls{nlp} formulation.

\begin{equation}
\mathbf{w} \coloneqq 
\begin{bmatrix}
    u \\
    w \\
    q \\
    \theta \\
    x \\
    z \\
    m \\
    \delta_{TPS} \\
    \delta_{e} \\
\end{bmatrix}
=
\begin{bmatrix}
    x_1 \\
    x_2 \\
    x_3 \\
    x_4 \\
    x_5 \\
    x_6 \\
    x_7 \\
    u_1 \\
    u_2 \\
\end{bmatrix}
\label{eq:Benchmark_States_Controls}
\end{equation}

\subsection{Constraints and bounds}
After the state space is set, the constraints can be defined, following second step on \gls{ocp} definition. At this point, previous remarks on constraint definition towards \gls{ocp} notation could be recovered. That means that constraints are described following the prestablished order, which starts by dynamic constraints. Since dynamic constraints have been defined in section \ref{sec:EOM}, they have to be transcribed using state space notation, which led to the following set of expressions.\footnote{From now on, sine operation $\sin()$ is denoted as $s$ and, cosine operation $\cos()$ is denoted as $c$, as clarified in nomenclature section This has been done for an easy readability of some equations and operations.}

\begin{equation}
\begin{split}
    \dot{x_1} ={} & \frac{1}{x_7} \Big[ T(x_1,x_2,x_6,u_1) + L(x_1,x_2,x_6,u_2) \cdot s\alpha(x_1,x_2) \\
    & - D(x_1,x_2,x_6,u_2) \cdot c\alpha(x_1,x_2) \Big] - x_3x_2 - g \cdot sx_4
\end{split}
\label{eq:Benchmark_EOM_1}
\end{equation}

\begin{equation}
\begin{split}
    \dot{x_2} ={} & g \cdot cx_4 - \frac{1}{x_7} \Big[ L(x_1,x_2,x_6,u_2) \cdot c\alpha(x_1,x_2) \\
    & + D(x_1,x_2,x_6,u_2) \cdot s\alpha(x_1,x_2) \Big] + x_3x_1 
\end{split}
\label{eq:Benchmark_EOM_2}
\end{equation}

\begin{equation}
\begin{split}
    \dot{x_3} ={} & \frac{1}{I_y} \Big[ C_m(u_2)\bar{q}(x_1,x_2,x_6)S\bar{c} \\
    & - T(x_1,x_2,x_6,u_1)\left[\Delta X_{CT}\cdot s \epsilon (x_1,x_2)+ \Delta Z_{CT}\cdot c \epsilon (x_1,x_2)\right]\Big]
\end{split}
\label{eq:Benchmark_EOM_3}
\end{equation}

\begin{equation}
    \dot{x_4} = x_3
\label{eq:Benchmark_EOM_4}
\end{equation}

\begin{equation}
    \dot{x_5} = x_1\cdot c x_4 + x_2\cdot s x_4
\label{eq:Benchmark_EOM_5}
\end{equation}

\begin{equation}
    \dot{x_6} = -x_1\cdot s x_4 + x_2\cdot c x_4
\label{eq:Benchmark_EOM_6}
\end{equation}

\begin{equation}
    \dot{x_7} = -SFC
\label{eq:Benchmark_EOM_7}
\end{equation}

The set of equations above can be summarised to a reduced version if RHS is expressed as fucntionals of the states and controls involved only.

\begin{equation}
    \mathbf{g_{dyn}} \coloneqq
\begin{bmatrix}
    \dot{x_1} -  f_1(x_1,x_2,x_3,x_4,x_6,x_7,u_1,u_2)\\
    \dot{x_2} -  f_2(x_1,x_2,x_3,x_4,x_6,x_7,u_2)\\
    \dot{x_3} - f_3(x_1,x_2,x_6,u_1,u_2)\\
    \dot{x_4} - f_4(x_3)\\
    \dot{x_5} - f_5(x_1,x_2,x_4)\\
    \dot{x_6} - f_6(x_1,x_2,x_4)\\
    \dot{x_7} -SFC\\
\end{bmatrix}
=
0
\label{eq:Benchmark_Dynamic_Constraints}
\end{equation}

Once dynamic constraints have been fully described, its time to set the path constraints and define them using state space notation. As said before, path constraints are the ones that could affect state and control trajectories during all time domain. For a level-flight cruise trajectory it is important to introduce a target speed ($V_{tp}$) and a target altitude ($h_{ref}$) as a constraints, see expressions below. It has to be pinpointed that flight path angle could be introduced as a constraint, but it has been chosen to introduce it later as a penalty into cost functional definition. Both constraints for target speed and target altitude are introduced using inequality expressions, so as not to be too restrictive with the problem -i.e the expressions has to introduce some margins with target constant values-.

\begin{equation}
    \begin{cases}
        0.9V_{tp} - \sqrt{(u - V_{x_w})^2 + (w - V_{z_w})^2} \leq 0 \\
        \sqrt{(u - V_{x_w})^2 + (w - V_{z_w})^2} - 1.1V_{tp} \leq 0
    \end{cases}
    \label{eq:Benchmark_Path_1}
\end{equation}

\begin{equation}
    \begin{cases}
        (h_{ref} - 3.0) + z \leq 0 \\
        -z - (h_{ref} + 3.0) \leq 0
    \end{cases}
    \label{eq:Benchmark_Path_2}
\end{equation}

Transcribing the equations using state space notation led to the following set of expressions.

\begin{equation}
    \mathbf{g_{path}} \coloneqq
    \begin{bmatrix}
        0.9V_{tp} - \sqrt{(x_1 - V_{x_w})^2 + (x_2 - V_{z_w})^2}\\
        \sqrt{(x_1 - V_{x_w})^2 + (x_2 - V_{z_w})^2} - 1.1V_{tp}\\
        (h_{ref} - 3.0) + x_6\\
        -x_6 - (h_{ref} + 3.0)
    \end{bmatrix}
    \leq 0
    \label{eq:Benchmark_Path_Constraints}
\end{equation}

Eventually, the simple bounds for each state and control variable have to be declared. The lower bounds and upper bounds are separated onto two different groups denoted by $lbx$ and $ubx$, respectively. They have been already denoted using state space notation.

\begin{equation}
    \mathbf{lbx} \coloneqq 
    \begin{bmatrix}
    u_{min} - x_1  \\
    w_{min} - x_2 \\
    - x_3 \\
    \theta_{min} - x_4 \\
    - x_5 \\
    z_{min} - x_6 \\
    BEM - x_7 \\  
    \end{bmatrix}
    \leq 0
    \label{eq:Benchmark_LBX}
\end{equation}

\begin{equation}
    \mathbf{ubx} \coloneqq 
    \begin{bmatrix}
    x_1 - u_{max}  \\
    x_2 - w_{max} \\
    x_3 \\
    x_4 - \theta_{max} \\
    x_5 \\
    x_6 - z_{max}\\
    x_7 - MTOM \\  
    \end{bmatrix}
    \leq 0
    \label{eq:Benchmark_UBX}
\end{equation}

\subsection{Cost functional}
Third phase of an \gls{ocp} definition is defining which variables have to be optimised and below which optimisation criteria. For this benchmark case, five different criteria have been chosen. 

\begin{itemize}
    \item Firstly, the minimisation of the flight path angle ($\gamma$) is introduced as a quadratic penalty in the objective function rather than a hard equality constraint. This formulation encourages the solver to converge on a solution where $\gamma \approx 0$, effectively approximating a level-flight trajectory. This approach was chosen because enforcing $\gamma = 0$ as an equality constraint can lead to numerical stiffness, significantly increasing the difficulty of finding a feasible solution. By using it as a penalty, the solver can more efficiently negotiate the trade-offs between system dynamics and path objectives, leading to a more robust and faster convergence toward the optimal solution. The expression introduced in the running cost is the following one, where $\bar{\gamma}$ is the normalised $\gamma$ term using $\gamma_{max}$.
    
    \begin{equation}
        J_{\gamma} \coloneqq \bar{\gamma}^2 = \left(\frac{\gamma}{\gamma_{max}}\right)^2 = \left(\frac{\theta - \alpha}{\gamma_{max}}\right)^2 = \left(\frac{x_4 - \alpha(x_1,x_2)}{\gamma_{max}}\right)^2
    \end{equation}

    \item Secondly, the minimisation of altitude error relative to the reference altitude ($h_{ref}$) is implemented as a quadratic penalty to promote a level-flight profile. Unlike the hard path constraints defined in \eqref{eq:Benchmark_Path_2}, which establish strict safety limits, the penalty allows for transient deviations during manoeuvres. This dual-layer approach—constraining the aircraft within a feasible corridor while penalizing deviations from the centerline—mimics real-world flight control laws. It allows the optimiser to prioritize control smoothness and actuator limits over rigid altitude tracking, resulting in a more physically realistic and numerically stable trajectory.

    \begin{equation}
        J_{h} \coloneqq (\bar{h} - 1)^2 = \left(\frac{h - h_{ref}}{h_{ref}}\right)^2 = \left(\frac{-x_6 - h_{ref}}{h_{ref}}\right)^2
    \end{equation}

    \item Thirdly, the minimisation of flight path angle rate ($\dot{\gamma}$) prevents the aircraft from rapidly oscillating up and down while forces the solver to find a solution that is not only level ($\gamma \approx 0$) but also steady ($\dot{\gamma} \approx 0$). The full derivation of the expression below can be found on annex \textcolor{red}{Annex:Gamma\_Dot}, since it involves partial derivatives and a large mathematical development.
    
    \begin{equation}
        J_{\dot{\gamma}} \coloneqq \left(\frac{\dot{w}u - \dot{u}w}{u^2 + w^2}\right)^2 / \dot{\gamma}_{max}^2 = \left(\frac{\dot{x_2}x_1 - \dot{x_1}x_2}{x_1^2 + x_2^2}\right)^2 / \dot{\gamma}_{max}^2
    \end{equation}

    \item Eventually, control variable rates minimisation have been introduced using a quadratic penalty, in each case, for preventing a bang-bang -rapidly control changing- behaviour, since it is a non-feasible behaviour in real systems. Furthermore, control rate minimisation is the normal criteria for real pilots when facing low-altitude flight manoeuvres, also known as "lazy pilot" implementation. This implementation basically reduces pilot workload while preventing overcontrol risks, as stated by Stengel in \cite{stengel1994optimal}. The cost expressions for both \gls{tps} and elevator deflection are the following ones, respectively -$t_{k+1}$ and $t_{k}$, refer to next and actual time-instants-.
    
    \begin{equation}
    J_{\delta_{TPS}} \coloneqq \frac{\left(\delta_{TPS}^{k+1} - \delta_{TPS}^{k}\right)^2}{\Delta t} = \frac{\left(u_1^{k+1} - u_1^{k}\right)^2}{\Delta t}
    \end{equation}
    \begin{equation}
    J_{\delta_{e}} \coloneqq \frac{\left(\delta_{e}^{k+1} - \delta_{e}^{k}\right)^2}{\delta_{e, max}^2\Delta t} = \frac{\left(u_2^{k+1} - u_2^{k}\right)^2}{\delta_{e,max}^2\Delta t}
    \end{equation}
\end{itemize}

\section{NLP Formulation and Implementation via CasADi}

\section{Results}

\section{Addition of the Free-End Condition}

