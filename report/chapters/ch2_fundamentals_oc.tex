\chapter{Fundamentals of Optimal Control and its Implementation}


\section{Introduction to Optimal Control Theory}
The \gls{oct} is a control field related branch where an objective function has to be optimised -most of the times, it has to be minimised- in order to find the control trajectory for a determined dynamical system. This theory is hystorically realted with calculus of variations, where optimal points -either maxima or minima- are found using lñittle variations in functions or functionals (\textcolor{red}{CITA}). This field of study, proposed by Isaac Newton, was developed as a solving approach for the brachistochrone problem, posed by Bernoulli in 1696. The brachistochrone curve is defined as the fastest descent path -most optimal path for minimising time- between two points A and B under a uniform gravitational field. Counterintuitively, it was found that the most optimal path was not but the cycloidal ramp, as can be seen on the image below, where ball has to roll under gravity only at the different paths, and the same grade shaded figures correspond to same time-instants.

In essence, the \gls{oct} responds the need to solve continuous time optimisation problems, as the once presented. Thus, the \gls{ocp} can be understood as a n-dimensional extension of the \gls{nlp} problem. The NLP problem can be defined as the minimisation of a given function subject to a different set of equations or inequations, called constraints.