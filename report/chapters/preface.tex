\chapter*{Preface}
\addcontentsline{toc}{chapter}{Preface}
This preface chapter includes a brief description of the scope of the project followed by the motivation that has led all the development. Then, the methodological approach is summarised, eventually followed by the report's structure.

\section*{Scope and Objectives}
\addcontentsline{toc}{section}{Scope and Objectives}
The scope of this project \underline{will} include:
\begin{itemize}
    \item Comprehensive review of trajectory optimisation problems and the approach solutions to the matter, identificating the limitations and research gaps.
    \item A theoretical comprehension and explanation of optimal control theory and restricted vertical flight trajectories modelisation.
    \item A mathematical formulation of firefighting manoeuvres under varying mass and heavy operational constraints.
    \item \gls{nlp} implementation and solution of the formulated problem above.
    \item A specialised water-drop trajectory planner algorithm.
\end{itemize}

Therefore, this project \underline{will not} include:
\begin{itemize}
    \item Physical design or manufacturing of the \gls{uav} and its payloads.
    \item Real-world flight testing or hardware tests.
    \item Modification of low-level autopilot firmware or inner-loop control laws.
    \item High-fidelity fluid modelling of the water-drop print or fire plume behaviour.
    \item Formal certification -e.g. RTCA DO-178C- or cybersecurity regulatory assessment.
    \item Full migration of the Python codebase to C/C++ or embedded deployment.
\end{itemize}

Eventually, the main objective of this thesis is to develop an algorithm capable of computing the optimal flight trajectory for a \gls{uav} operating in firefighting missions. The algorithm is designed to generate restricted trajectories within a vertical plane while considering multiple operational constraints. Specifically, the algorithm aims to determine the descent path during the water-drop manoeuvre and the subsequent ascent trajectory, ensuring feasability and safety. This thesis and project has been developed in collaboration with Singular Aircraft, as part of an extracurricular internship agreement along with other tasks.

\section*{Motivation and Justification}
\addcontentsline{toc}{section}{Motivation and Justification}
In recent years, the increasing occurrence and intensity of wildfires have highlighted the urgent need for more effective and autonomous aerial firefighting solutions \cite{FAOFireStats2023}. From a global perspective, enhancing the autonomy and operational efficiency of \gls{uas} for emergency response represents a major step toward safer and more sustainable fire suppression operations. In particular, fixed-wing \gls{uav}s offer significant advantages over rotary-wing systems, including greater range, endurance, and payload capacity, making them suitable for long-duration firefighting missions in hazardous environments \cite{UAVFireSuppression2022}.

This project is developed within the framework of an extracurricular internship at Singular Aircraft, an aerospace company founded by Luis Carrillo with over twelve years of experience in the design, manufacture, operation, and certification of \gls{rpa} and \gls{uav}. The company has designed and fully developed the Flyox, currently the largest civilian fixed-wing \gls{uav} designed for firefighting, humanitarian aid, and surveillance missions \cite{SingularWeb}. Several previous projects and studies carried out within the company have focused on the improvement of the Flyox's flight systems, payload integration, and mission control capabilities. Building upon those developments, this thesis contributes to the new generation of the Flyox platform, which aims to introduce advanced levels of autonomy for critical mission phases.

Specifically, this new generation seeks to enable the aircraft to perform the water-drop manoeuvre autonomously, reducing operator workload and enabling missions under low-visibility or high-risk conditions. To achieve this, the aircraft must be equipped with a trajectory computation algorithm capable of planning and re-planning the discharge trajectory in real time, considering aircraft dynamics, environmental constraints, and safety margins, without human validation. This capability represents a fundamental step toward full operational autonomy in firefighting \gls{uas}.

The current work focuses on developing such an algorithm through optimal control and numerical optimisation techniques. The approach aims to fill existing gaps in trajectory generation software for \gls{uas} by integrating non-standard operational constraints such as \gls{dem}, \gls{nfz}, atmospheric conditions, and aircraft operational limits. Although not all these constraints will be implemented in the present study, they define the broader framework in which this research is embedded.

From a critical standpoint, the proposed approach offers clear advantages: it enables a higher degree of autonomy, improved safety by reducing human involvement in dangerous environments, and adaptability to real-time mission changes. However, potential limitations include the computational cost of real-time optimisation, the need for accurate environmental data, and the complexity of integrating such systems into certified aircraft architectures. Despite these challenges, this research represents an essential step toward achieving the next level of autonomy in firefighting \gls{uas} operations.

From my viewpoint as the author of the thesis and the project, I was thrilled to be able to contribute to a real project, such as Singular Aircraft and the guidance of manoeuvres in the Flyox \gls{uav}, as well as to learn about a subject that was new to me and develop my skills in it.

\section*{Methodological Approach}
\addcontentsline{toc}{section}{Methodological Approach}
The methodological framework employed to solve the firefighting trajectory optimisation problem has been a combination of \gls{oct} and \gls{nlp}. Specifically, the physical system has been modelled through differential-algebraic equations, which are subsequently transformed into \gls{nlp} problem via the direct transcription method and a trapezoidal integration scheme. Then, the \gls{nlp} have been implemented and solved using a CasADi framework, including the definition of objective functions, constraint handling, and solver configurations. 

\section*{Document Structure}
\addcontentsline{toc}{section}{Document Structure}
This document is structured in seven (7) different chapters. Firstly, in the State-of-the-art chapter, some research has been done in matters of trajectory optimisation solutions until actual limitations and research gaps have been identified. Then, the second chapter introduces the reader to \gls{oct} through characteristic equations and definitions until reaching the \gls{nlp} implementation using CasADi, a dedicated optimisation library. Third chapter contains a level-cruise flight trajectory defined as an \gls{ocp} and solved using \gls{nlp} implementation. The chapter describes and solves two different problems, one without and one with the free-end condition. In fourth place, the whole firefighting manoeuvre has been described and modelled using \gls{oc} notation. Then, some simulations have been done and the results have been presented on Chapter 5. In sixth place, a budget and enviromental impact analysis of this project has been included. Eventually, the main contributions have been summarised on Conlusions dedicated chapter. The bibliography and all appendices have been added right after the concluding remarks.
