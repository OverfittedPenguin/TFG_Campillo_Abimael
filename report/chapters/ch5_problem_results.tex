\chapter{Firefighting Manoeuvre: Results}
Next chapter includes a comprehensive analysis of the results obtained for different simulation performed. Since the firefighting manoeuvre is a multi-parameter optimisation problem, only a few set of parameter variants have been tried. Then, a full review has been done and some remarks of the results have been added.

\section{Simulations}
In order to compare a few things among the different simulations, it is important to describe a base set of parameters to stablish later comparisons. The following table summarises the value of the parameters set on \textit{Simulation.json}. The aircraft used for this simulation and the following ones has been the \gls{uav} Flyox, which parameters can be seen on appendix \ref{app:AircraftParameters} table \ref{tab:Flyox_Parameters}. In addition, all simulations have been performed with 150 collocation point -or nodes-, wich is enough to capture water-discharge stage.

SIMULATION PARAMETERS TAB

\begin{figure}[H]
    \centering
    \includesvg[width=\textwidth]{images/manoeuvre/CASE0/STATES} 
    \caption{State trajectories, base case. States included are velocities, angles, mass and \gls{uav} trajectory. Own source.}
    \label{fig:Base_STATES}
\end{figure}

\begin{figure}[H]
    \centering
    \includesvg[width=\textwidth]{images/manoeuvre/CASE0/CONTROLS} 
    \caption{Control trajectories, base case. Controls included are \gls{tps} and elevator. Own source.}
    \label{fig:Base_CONTROLS}
\end{figure}

Once all parameters have been properly set, the state and control trajectories are the following ones. As can be seen on figure \ref{fig:Base_STATES}, the water-discharge has been modelled as a discontinuity for this base case -indeed, it is a one-second discharge but since \gls{PM} is two-hundred (200) kg, discharge can be suddenly interpreted-. Descent and climb stages are almost continuous, as can be seen by nearly constant value on flight path angle ($\gamma$) or directly on the \gls{uav} trajectory. In terms of velocity, its profile may be suprisingly, but it denotes a key aspect of Flyox behaviour: longitudinal movements are closely coupled, as can be seen on velocities subgraph -difference between body velocity along $x_b$-axis $u$ and velocity modulus $V$-. Thus, the aircraft tends to align the descent or ascent path, provoking changes on velocity's module. In terms of controls, discontinuities are unavoidable because they correspond with the jump between stages; nevertheless, there are discontinuities affordable to a control feedback loop proper of an autopilot on an \gls{uav}. They will not be further mentioned on the report. Moreover, control seems smooth in each stage and they are far from the operational limits. Essentially, controls reach a trim point in each stage, more or less.

Following figure shows the evolution of cost objective per iteration for each stage. As can be seen, the demanding stages in terms of cost are the most restrictive ones: descent and climb stages, specifically descent one. This means that some constraints have been breached or nearly breached -i.e the solution found in the iteration which cost is highest is not feasible, constraints or penalties have been not fulfilled and the control turns agressive to avoid constraint breach-.

\begin{figure}[H]
    \centering
    \includesvg[width=\textwidth]{images/manoeuvre/CASE0/COST} 
    \caption{Cost objective per iteration, base case. Each stage cost has been added. Own source.}
    \label{fig:Base_COST}
\end{figure}

\subsection{Case 1: continuous water-discharge}
This case has been done using the same parameters that have been indicated on table \ref{tab:FFM_Parameters} except for the discharge time ($t_d$), which has been incremented to perform a more continuous and controlled discharge above TP. The discharge time has been set to $t_d = 7.2\ s$, which is near to firefighting controlled drops -it has been specified by Singular Aircraft engineers-. 

\begin{figure}[H]
    \centering
    \includesvg[width=\textwidth]{images/manoeuvre/CASE1/STATES} 
    \caption{State trajectories, case 1. States included are velocities, angles, mass and \gls{uav} trajectory. Own source.}
    \label{fig:Case1_STATES}
\end{figure}

\begin{figure}[H]
    \centering
    \includesvg[width=\textwidth]{images/manoeuvre/CASE1/CONTROLS} 
    \caption{Control trajectories, case 1. Controls included are \gls{tps} and elevator. Own source.}
    \label{fig:Case1_CONTROLS}
\end{figure}

As can be seen on previous figures, there are no noticeable changes on state trajectories. However, there is a minor change in term of controls. Since the gates are open for a longer period of time, control corrections are soft and smooth than in the previous case. The following figure shows a comparison between the second stage control from base case -top- and this case -bottom-. 

\begin{figure}[H]
    \centering
    \begin{subfigure}[b]{\textwidth}
        \centering
        \includesvg[width=\textwidth]{images/manoeuvre/CASE0/STG2/CONTROLS}
        \caption{Base case, discontinuous discharge.}
    \end{subfigure}
    
    \vspace{0.25cm}
    
    \begin{subfigure}[b]{\textwidth}
        \centering
        \includesvg[width=\textwidth]{images/manoeuvre/CASE1/STG2/CONTROLS}
        \caption{Case 1, continuous discharge.}
    \end{subfigure}
    \caption{Comparison between control trajectories during a two different discharge types. Own source.}
    \label{fig:comparison_states}
\end{figure}

\subsection{Case 2: restrictive distances and end-time penalties}

\subsection{Case 3: controls and, desired attitudes not penalised}